\documentclass[crop=false]{standalone}

\usepackage{dissertationstyle}

\begin{document}
  \begin{tabular}{ll}
    Candidate Number: &\textbf{2338A} \\
    Title: &\textbf{Identifying Pianists through Audio of a Performance} \\
    Examination: &\textbf{Computer Science Tripos -- Part II} \\
    Year: &\textbf{2022} \\
    Word Count: &\textbf{11851}\footnotemark \\
    Code Line Count: &\textbf{2332}\footnotemark \\
    Project Originator: &\textbf{Dr John K. Fawcett} \\
    Project Supervisor: &\textbf{Dr John K. Fawcett}
  \end{tabular}

  \section*{Original Aims of the Project}

  The original aims of the project were to implement a system that was intended to be able to identify who the performer of a piece of piano music was, based on the audio signal of that performance and of previous performances of the same piece. This system would use 5 metrics, and the project's aim was to evaluate how useful each of these 5 metrics are, and under what conditions they are useful.

  \section*{Work Completed}
  I implemented each of the initially proposed metrics: tempo variation over time, dynamics over time, chroma vector extraction, note offsets, and timbre extraction. I also implemented a data synthesis module that allowed me to synthesise piano data to test each of these metrics on before I had completed real-world data collection. Furthermore, I implemented various transformations that could be applied to the data to better simulate performance of the system under less forgiving conditions.

  \section*{Special Difficulties}
  None.

  \footnotetext[1]{Calculated using \texttt{texcount} (found at \url{https://app.uio.no/ifi/texcount/})}
  \footnotetext[2]{Lines of Python, counted using \texttt{find . -name "*.py" | xargs wc -l}}



\end{document}
